% \iffalse meta-comment
%
% Copyright (C) 2019 by Daniel Bosk <dbosk@kth.se>
% -------------------------------------------------------
% 
% This file may be distributed and/or modified under the
% conditions of the LaTeX Project Public License, either version 1.3
% of this license or (at your option) any later version.
% The latest version of this license is in:
%
%    http://www.latex-project.org/lppl.txt
%
% and version 1.3 or later is part of all distributions of LaTeX 
% version 2005/12/01 or later.
%
% \fi
%
% \iffalse
%<*driver>
\ProvidesFile{didactic.dtx}
%</driver>
%<package>\NeedsTeXFormat{LaTeX2e}
%<package>\ProvidesPackage{didactic}
%<*package>
    [2019/01/29 v1.0 beamer-didactic]
%</package>
%<package>\RequirePackage{xparse}
%
%<*driver>
\documentclass{ltxdoc}
\usepackage[utf8]{inputenc}
\usepackage[british]{babel}
\usepackage{didactic}
\EnableCrossrefs         
\CodelineIndex
\RecordChanges
\begin{document}
  \DocInput{didactic.dtx}
  \PrintChanges
  %\PrintIndex
\end{document}
%</driver>
% \fi
%
% \CheckSum{0}
%
% \CharacterTable
%  {Upper-case    \A\B\C\D\E\F\G\H\I\J\K\L\M\N\O\P\Q\R\S\T\U\V\W\X\Y\Z
%   Lower-case    \a\b\c\d\e\f\g\h\i\j\k\l\m\n\o\p\q\r\s\t\u\v\w\x\y\z
%   Digits        \0\1\2\3\4\5\6\7\8\9
%   Exclamation   \!     Double quote  \"     Hash (number) \#
%   Dollar        \$     Percent       \%     Ampersand     \&
%   Acute accent  \'     Left paren    \(     Right paren   \)
%   Asterisk      \*     Plus          \+     Comma         \,
%   Minus         \-     Point         \.     Solidus       \/
%   Colon         \:     Semicolon     \;     Less than     \<
%   Equals        \=     Greater than  \>     Question mark \?
%   Commercial at \@     Left bracket  \[     Backslash     \\
%   Right bracket \]     Circumflex    \^     Underscore    \_
%   Grave accent  \`     Left brace    \{     Vertical bar  \|
%   Right brace   \}     Tilde         \~}
%
%
% \changes{v1.0}{2019/01/29}{Initial version}
%
% \GetFileInfo{didactic.dtx}
%
% \DoNotIndex{\newcommand,\newenvironment}
% 
%
% \title{The \textsf{didactic} package\thanks{This document
%   corresponds to \textsf{didactic}~\fileversion, dated \filedate.}}
% \author{Daniel Bosk\\\texttt{daniel@bosk.se}}
%
% \maketitle
%
% \section{Introduction}
%
% This package introduces some environments that are useful for writing 
% teaching material.
%
% \section{Usage}
%
% \DescribeEnv{assumption}
% \DescribeEnv{idea}
% \DescribeEnv{question}
% \DescribeEnv{exercise}
% \DescribeEnv{remark}
% These environments provides wrappers around Beamer's |block| environment.
% Each of them provides a block with an appropriate title and optional subtitle 
% and is of a suitable colour.
%
% \DescribeEnv{lightblock}
% \DescribeEnv{darkblock}
% We provide two shades of boxes, light coloured and dark coloured boxes.
% These can be used to create boxes of various colours easily.
% These two environments take two mandatory arguments: the first one is a 
% colour, the second is the title.
%
% \DescribeEnv{blackblock}
% \DescribeEnv{whiteblock}
% \DescribeEnv{bwblock}
% For instance, we provide a black block, white block and a black-white block.
% These take only one mandatory argument: a title.
%
% \DescribeEnv{redblock}
% \DescribeEnv{blueblock}
% \DescribeEnv{purpleblock}
% \DescribeEnv{greenblock}
% \DescribeEnv{yellowblock}
% \DescribeEnv{orangeblock}
% We also provide a few predefined blocks in various common colours.
% They all take a title as mandatory argument.
%
% \StopEventually{}
%
% \section{Implementation}
%
% \begin{environment}{assumption}
% \begin{environment}{idea}
% \begin{environment}{question}
% \begin{environment}{exercise}
% \begin{environment}{remark}
% \begin{environment}{summary}
% These environments wraps Beamer's |block| environment: it sets a default 
% title and colour with an optional name.
%    \begin{macrocode}
\ProvideDocumentEnvironment{assumption}{o}{%
  \IfValueTF{#1}{%
    \begin{block}{Assumption: #1}
  }{%
    \begin{block}{Assumption}
  }
}{%
  \end{block}
}

\ProvideDocumentEnvironment{protocol}{o}{%
  \IfValueTF{#1}{%
    \begin{block}{Protocol: #1}
  }{%
    \begin{block}{Protocol}
  }
}{%
  \end{block}
}

\ProvideDocumentEnvironment{remark}{o}{%
  \IfValueTF{#1}{%
    \begin{alertblock}{Note: #1}
  }{%
    \begin{alertblock}{Note}
  }
}{%
  \end{alertblock}
}

\ProvideDocumentEnvironment{idea}{o}{%
  \IfValueTF{#1}{%
    \begin{block}{Idea: #1}
  }{%
    \begin{block}{Idea}
  }
}{%
  \end{block}
}

\ProvideDocumentEnvironment{question}{o}{%
  \setbeamercolor{block body}{bg=orange!15,fg=black}
  \setbeamercolor{block title}{bg=orange,fg=white}
  \setbeamercolor{local structure}{fg=orange}
  \IfValueTF{#1}{%
    \begin{block}{Question: #1}
  }{%
    \begin{block}{Question}
  }
}{%
  \end{block}
}

\ProvideDocumentEnvironment{exercise}{o}{%
  \setbeamercolor{block body}{bg=yellow!10,fg=black}
  \setbeamercolor{block title}{bg=yellow,fg=black}
  \setbeamercolor{local structure}{fg=yellow}
  \IfValueTF{#1}{%
    \begin{block}{Exercise: #1}
  }{%
    \begin{block}{Exercise}
  }
}{%
  \end{block}
}

\ProvideDocumentEnvironment{summary}{o}{%
  \IfValueTF{#1}{%
    \begin{block}{Summary: #1}
  }{%
    \begin{block}{Summary}
  }
}{%
  \end{block}
}
%    \end{macrocode}
% \end{environment}
% \end{environment}
% \end{environment}
% \end{environment}
% \end{environment}
% \end{environment}
%
% \begin{environment}{lightblock}
% \begin{environment}{darkblock}
% \begin{environment}{blackblock}
% \begin{environment}{whiteblock}
% \begin{environment}{bwblock}
% \begin{environment}{redblock}
% \begin{environment}{blueblock}
% \begin{environment}{purpleblock}
% \begin{environment}{greenblock}
% \begin{environment}{yellowblock}
% \begin{environment}{orangeblock}
% Now we have the coloured blocks.
%    \begin{macrocode}
\ProvideDocumentEnvironment{lightblock}{mm}{%
  \setbeamercolor{block body}{bg=#1!10,fg=black}
  \setbeamercolor{block title}{bg=#1,fg=black}
  \setbeamercolor{local structure}{fg=#1}
  \begin{block}{#2}
}{%
  \end{block}
}
\ProvideDocumentEnvironment{darkblock}{mm}{%
  \setbeamercolor{block body}{bg=#1!10,fg=black}
  \setbeamercolor{block title}{bg=#1,fg=white}
  \setbeamercolor{local structure}{fg=#1}
  \begin{block}{#2}
}{%
  \end{block}
}

\ProvideDocumentEnvironment{blackblock}{m}
{\begin{darkblock}{black}{#1}}
{\end{darkblock}}
\ProvideDocumentEnvironment{whiteblock}{m}{%
  \setbeamercolor{block body}{bg=white!10,fg=black}
  \setbeamercolor{block title}{bg=white,fg=black}
  \setbeamercolor{local structure}{fg=black}
  \begin{block}{#1}
}{%
  \end{block}
}
\ProvideDocumentEnvironment{bwblock}{m}{%
  \setbeamercolor{block body}{bg=white!10,fg=black}
  \setbeamercolor{block title}{bg=black,fg=white}
  \setbeamercolor{local structure}{fg=black}
  \begin{block}{#1}
}{%
  \end{block}
}

\ProvideDocumentEnvironment{redblock}{m}
{\begin{darkblock}{red}{#1}}
{\end{darkblock}}
\ProvideDocumentEnvironment{blueblock}{m}
{\begin{darkblock}{blue}{#1}}
{\end{darkblock}}
\ProvideDocumentEnvironment{purpleblock}{m}
{\begin{darkblock}{purple}{#1}}
{\end{darkblock}}

\ProvideDocumentEnvironment{greenblock}{m}
{\begin{lightblock}{green}{#1}}
{\end{lightblock}}
\ProvideDocumentEnvironment{yellowblock}{m}
{\begin{lightblock}{yellow}{#1}}
{\end{lightblock}}
\ProvideDocumentEnvironment{orangeblock}{m}
{\begin{lightblock}{orange}{#1}}
{\end{lightblock}}
%    \end{macrocode}
% \end{environment}
% \end{environment}
% \end{environment}
% \end{environment}
% \end{environment}
% \end{environment}
% \end{environment}
% \end{environment}
% \end{environment}
% \end{environment}
% \end{environment}
%
% \Finale
\endinput
