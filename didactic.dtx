% \iffalse meta-comment
%
% Copyright (C) 2019--2020, 2022, 2024 by Daniel Bosk <daniel@bosk.se>
% -------------------------------------------------------
% 
% This file may be distributed and/or modified under the
% conditions of the LaTeX Project Public License, either version 1.3c
% of this license or (at your option) any later version.
% The latest version of this license is in:
%
%    http://www.latex-project.org/lppl.txt
%
% and version 1.3 or later is part of all distributions of LaTeX 
% version 2005/12/01 or later.
%
% \fi
%
% \iffalse
%<*driver>
\ProvidesFile{didactic.dtx}
%</driver>
%<package>\NeedsTeXFormat{LaTeX2e}
%<package>\ProvidesPackage{didactic}
%<*package>
  [2024/08/16 v2.3 didactic]
%</package>
%<package>\RequirePackage{xparse}
%<package>\RequirePackage{xkeyval}
%<package>\RequirePackage{xstring}
%<package>\RequirePackage{etoolbox}
%<package>\RequirePackage{pythontex}
%<package>\RequirePackage{minted}
%<package>\RequirePackage{babel}
%<package>\RequirePackage{translations}
%<package>\RequirePackage{amsthm}
%<package>\RequirePackage{thmtools}
%<package>\PassOptionsToPackage{unq}{unique}
%<package>\RequirePackage{unique}
%
%<*driver>
\documentclass{article}
\usepackage{doc}
\NewDocElement[printtype={opt.}]{Opt}{option}
\usepackage[utf8]{inputenc}
\usepackage[swedish,british]{babel}
\usepackage{pythontex}
\usepackage[notheorems]{beamerarticle}
\usepackage[marginparmargin=right]{didactic}
\usepackage{shortvrb}
\MakeShortVerb{\|}
\usepackage{lipsum}
\usepackage{hyperref}
\usepackage{cleveref}
\EnableCrossrefs
\CodelineIndex
\RecordChanges
\begin{document}
  \DocInput{didactic.dtx}
  \PrintChanges
  %\PrintIndex
\end{document}
%</driver>
% \fi
%
% \CheckSum{0}
%
% \CharacterTable
%  {Upper-case    \A\B\C\D\E\F\G\H\I\J\K\L\M\N\O\P\Q\R\S\T\U\V\W\X\Y\Z
%   Lower-case    \a\b\c\d\e\f\g\h\i\j\k\l\m\n\o\p\q\r\s\t\u\v\w\x\y\z
%   Digits        \0\1\2\3\4\5\6\7\8\9
%   Exclamation   \!     Double quote  \"     Hash (number) \#
%   Dollar        \$     Percent       \%     Ampersand     \&
%   Acute accent  \'     Left paren    \(     Right paren   \)
%   Asterisk      \*     Plus          \+     Comma         \,
%   Minus         \-     Point         \.     Solidus       \/
%   Colon         \:     Semicolon     \;     Less than     \<
%   Equals        \=     Greater than  \>     Question mark \?
%   Commercial at \@     Left bracket  \[     Backslash     \\
%   Right bracket \]     Circumflex    \^     Underscore    \_
%   Grave accent  \`     Left brace    \{     Vertical bar  \|
%   Right brace   \}     Tilde         \~}
%
%
% \changes{v1.0}{2019/01/29}{Initial version}
% \changes{v1.1}{2019/03/26}{Adds summary environment}
% \changes{v1.2}{2019/10/26}{Adds lightblock, darkblock and coloured blocks}
% \changes{v1.3}{2019/01/29}{Renames to didactic; adds tools for code examples}
% \changes{v1.4}{2023/03/20}{%
%   Adds settings for memoir and beamer, contrasting code examples, 
%   translations.
% }
% \changes{v1.5}{2023/03/25}{Adds |\ProvideSemanticEnv|}
% \changes{v1.6}{2024/04/07}{%
%   Sets beamer bibliography item correctly.
%   Adds |fullwidth| environment, handles margins better.
%   Adds |sidecaption| environment and |\flushscap| command.
% }
% \changes{v1.7}{2024/04/08}{%
%   Adds |\NewNoteType| that improves |\newnotetype| with toggles.
% }
% \changes{v1.8}{2024/05/21}{%
%   Adds support for XeLaTeX.
% }
% \changes{v1.9}{2024/07/15}{%
%   Adds PythonTeX functions.
%   Typesets outputs using |minted| to get highlighted lines there too.
%   Only use |arthangnum| chapterstyle if |memoir| article mode.
% }
% \changes{v1.10}{2024/07/15}{%
%   Makes |csquotes| use |autocite| for citations.
% }
% \changes{v1.11}{2024/07/15}{%
%   Adds use of |blockmargin| and |unblockmargin| (from |marginfix|) for 
%   full-width elements.
%   Fixes |sidecaption|.
% }
% \changes{v2.0}{2024/07/15}{%
%   Makes starred versions of |\textbytext| (and related) that uses full width 
%   and blocks margin.
%   The default is to not use the full width (this is a breaking change).
%   In line with this, removes the |blockmargin| keyword argument to 
%   |didactic_mint| PythonTeX function.
%   This is merged into the |fullwidth| keyword argument.
% }
% \changes{v2.1}{2024/07/23}{%
%   Fixes the |\textbytext| command so that it can handle minted code.
% }
% \changes{v2.2}{2024/08/10}{%
%   Adds |didactic_shell| and |didactic_help| PythonTeX functions.
% }
% \changes{v2.3}{2024/08/16}{%
%   Turns off hangnum by default.
% }
%
% \GetFileInfo{didactic.dtx}
%
% \DoNotIndex{\newcommand,\newenvironment}
% 
%
% \title{The \textsf{didactic} package\thanks{This document
%   corresponds to \textsf{didactic}~\fileversion, dated \filedate.
%   Licensed under the terms of LPPL, version 1.3c or later.}}
% \author{Daniel Bosk\\\texttt{daniel@bosk.se}}
%
% \maketitle
%
% \tableofcontents
%
% \section{Introduction}
%
% This package introduces some environments that are useful for writing 
% teaching material.
% It also provides a few settings for |beamer| and |memoir| to make the 
% resulting document more readable.
%
% \section{Usage}
%
% The package automatically detects if |beamer| or |memoir| is loaded and 
% customizes the document accordingly.
% \DescribeOpt{nobeamer}
% \DescribeOpt{noarticle}
% \DescribeOpt{nomemoir}
% That can be prevented, however, by using the |nobeamer| or |nomemoir| 
% options.
% The |noarticle| is to prevent the article versions of the (|beamer|) 
% environments to be defined.
%
% \DescribeOpt{notheorems}
% The package also provides a few environments for |beamer| and |article| mode.
% Particularly, it can define theorems and definitions, which might be defined 
% also by other packages.
% To prevent this, use the |notheorems| option.
%
% \DescribeOpt{inner}
% \DescribeOpt{outer}
% \DescribeOpt{top}
% \DescribeOpt{bottom}
% \DescribeOpt{marginparmargin}
% When the package customizes |memoir| it sets the inner and outer margins to 
% fit margin notes better.
% The |inner| and |outer| options can be used to set the inner and outer 
% margins to be used.
% The |top| and |bottom| options can be used to set the top and bottom margins.
%
% The |marginparmargin| is passed on to the |memoir| class and sets which 
% margin to use for the margin notes.
%
% \subsection{Beamer blocks and semantic environments for papers}
%
% \DescribeEnv{assumption}
% \DescribeEnv{idea}
% \DescribeEnv{question}
% \DescribeEnv{exercise}
% \DescribeEnv{activity}
% \DescribeEnv{remark}
% These environments provides wrappers around Beamer's |block| environment.
% The idea is to complement the blocks provided by Beamer, such as |theorem|, 
% |definition| and |example|, with some more useful blocks.
% Each of them provides a block with an appropriate title and optional subtitle 
% and is of a suitable colour.
% For instance, consider the following.
% \begin{idea}[These blocks]\label{theidea}
  The idea of these blocks is to be able to use them in both slides and text.
  This way, we can focus on writing just one version of the content.
\end{idea}

% That can be produced by the following code.
% \inputminted[linenos]{latex}{idea.tex}
%
% \DescribeMacro{\ProvideSemanticEnv}
% We can create new such block environments, that work both with |beamer| and
% without, by using the |\ProvideSemanticEnv| command.
% This command takes five arguments:
% \begin{enumerate}
% \item the name of the environment,
% \item the name of the block environment to use (optional),
% \item the title of the block,
% \item options to pass to |thmtools| (optional),
% \item the key of the translation of the refname of the block,
% \item the key of the translation of the refnames of the block.
% \item the key of the translation of the Refname of the block,
% \item the key of the translation of the Refnames of the block.
% \end{enumerate}
% For example, the |idea| block above was created by the following code:
% \inputminted{latex}{ProvideSemanticEnv.tex}
%
% \DescribeEnv{lightblock}
% \DescribeEnv{darkblock}
% We provide two shades of boxes, light coloured and dark coloured boxes.
% These can be used to create boxes of various colours easily.
% These two environments take two mandatory arguments: the first one is a 
% colour, the second is the title.
%
% In the text (not slides) the blocks are not coloured.
% For instance, consider the following.
% \begin{lightblock}{green}{Title}
  This is the body.
\end{lightblock}

% That can be produced by the following code.
% \inputminted[linenos]{latex}{lightblock.tex}
%
% \DescribeEnv{blackblock}
% \DescribeEnv{whiteblock}
% \DescribeEnv{bwblock}
% For instance, we provide a black block, white block and a black-white block.
% These take only one mandatory argument: a title.
%
% \DescribeEnv{redblock}
% \DescribeEnv{blueblock}
% \DescribeEnv{purpleblock}
% \DescribeEnv{greenblock}
% \DescribeEnv{yellowblock}
% \DescribeEnv{orangeblock}
% We also provide a few predefined blocks in various common colours.
% They all take a title as mandatory argument.
%
% \subsection{Adding margin notes of different kinds}
%
% \DescribeMacro{\indentedmarginpar}
% We want to be able to put quite some content in the margins.
% For this reason we want to have indented paragraphs in the margin notes.
% We can use the |\indentedmarginpar{text}| command to do this.
% For instance, consider the output in the margin.
% \indentedmarginpar{%
%   This is an example of an indented margin paragraph.
%
%   It can contain multiple paragraphs.
% }
%
% \DescribeMacro{\newnotetype}
% \DescribeMacro{\ltnote}
% We also want to be able to use titled notes of different kinds.
% We can use the |\newnotetype[fmtcmd]{type}{title}| command to define a new 
% type of note.
% For instance |\ltnote| is defined as
% \begin{center}
% |\newnotetype{\ltnote}{\GetTranslationWarn{Learning theory}}|.
% \end{center}
% \ltnote{This is an example of a learning theory note.}
%
% We can also change the formatting of the title.
% |\newnotetype[\textbf]{\bfnote}{Bold note}| and
% |\bfnote{This is a bold note.}| gives us the note in the margin.
% \newnotetype[\textbf]{\bfnote}{Bold note}
% \bfnote{This is a bold note.}
% \ltnote{Another example of a learning theory note.}
%
% \DescribeMacro{\NewNoteType}
% We also have |\NewNoteType| which is almost synonymous to |\newnotetype|.
% The difference is that |\NewNoteType| doesn't take a command as argument, but 
% just a name:
% |\NewNoteType{ltnote}{Learning theory}| instead.
% Then we also get |\ltnoteoff| to turn off the note and |\ltnoteon| to turn it 
% back on.
%
% Notes are off by default in |beamer|, but on by default otherwise.
%
% \subsection{Tools for code examples}
%
% We also provide two commands for working with code examples: |lstexample| and 
% |runpython|.
%
% \DescribeMacro{\lstexample}
% The |\lstexample| command allows us to typeset the source code of the 
% example.
% For instance, |\lstexample{python}{hello.py}| produces the following:
% \lstexample{python}{hello.py}
%
% We can also pass optional arguments directly to |minted| that is used to 
% typeset the code.
% \begin{example}
% |\lstexample[linenos,highlightlines=4]{python}{hello.py}| produces the 
% following.
% \lstexample[linenos,highlightlines=4]{python}{hello.py}
% \end{example}
%
% \DescribeMacro{\runpython}
% We can also run the example code and include its output using the 
% |\runpython| macro.
% \begin{example}
% For instance, |\runpython[highlightlines=1]{hello.py}| produces the 
% following.
% \runpython[highlightlines=1]{hello.py}
% \end{example}
%
% \DescribeMacro{\codebycode}
% The |\codebycode| command is simply two |\lstexample| commands side by side:
% |\codebycode[opt1]{lang1}{file1}[opt2]{lang2}{file2}|.
%
% \begin{example}
% For instance,
% |\codebycode[breaklines,linenos]{python}{hello.py}|
%            |[breaklines,linenos,highlightlines=4]{python}{hello.py}|
% yields the following.
% \codebycode
%   [breaklines,linenos]{python}{hello.py}
%   [breaklines,linenos,highlightlines=4]{python}{hello.py}
% \end{example}
%
% \begin{example}
% We can also use the full width of the page with the starred version.
% |\codebycode*[breaklines,linenos]{python}{hello.py}|
%            |[breaklines,linenos,highlightlines=4]{python}{hello.py}|
% yields the following.
% \codebycode*
%   [breaklines,linenos]{python}{hello.py}
%   [breaklines,linenos,highlightlines=4]{python}{hello.py}
% \end{example}
%
% \DescribeMacro{\runbyrun}
% We can do similarly with running code.
%
% \begin{example}
% We can also put the output side by side using 
% |\runbyrun{hello.py}{hello.py}|.
% \runbyrun{hello.py}{hello.py}
% \end{example}
%
% Those commands are implemented using a set of Python functions that can be 
% accessed through PythonTeX.
% See \cref{PythonTeXdidactic} for details.
%
% \StopEventually{}
%
% \section{Implementation}
%
% Let's start with the options.
%
% \subsection{Options}
%
% We have a few negative options, that is, if specified we don't want to do 
% them.
% This means that we'll need to use conditionals.
% So each option definition consists first of a |\newif| then setting it to 
% true, and finnally, a |\DeclareOption|.
% The body of the |\DeclareOption| then sets the |\if| to false.
% After all the options are declared we'll use |\ProcessOptions|.
% After that we can have all code actually doing the work, all wrapped in |\if| 
% statements.
%
% \begin{option}{nobeamer}
% \begin{option}{nobeamertheme}
% \begin{option}{noarticle}
% This option is used to disable the |beamer| specific parts of the package.
% Meaning that we don't customize |beamer| even when we find that it is loaded.
%    \begin{macrocode}
\newif\if@didactic@beamer
\@ifclassloaded{beamer}{\@didactic@beamertrue}{\@didactic@beamerfalse}
\DeclareOptionX{nobeamer}{\@didactic@beamerfalse}
\newif\if@didactic@article
\@ifclassloaded{beamer}{\@didactic@articlefalse}{\@didactic@articletrue}
\newif\if@didactic@beamertheme
\@didactic@beamerthemetrue
\DeclareOptionX{nobeamertheme}{\@didactic@beamerthemefalse}
\DeclareOptionX{noarticle}{\@didactic@articlefalse}
%    \end{macrocode}
% \end{option}
% \end{option}
% \end{option}
%
% \begin{option}{nomemoir}
% This option is used to disable the |memoir| specific parts of the package.
% Meaning that we don't customize |memoir| even when we find that it is loaded.
%    \begin{macrocode}
\newif\if@didactic@memoir
\@ifclassloaded{memoir}{\@didactic@memoirtrue}{\@didactic@memoirfalse}
\DeclareOptionX{nomemoir}{\@didactic@memoirfalse}
%    \end{macrocode}
% \end{option}
%
% \begin{option}{notheorems}
% This option is used to disable defining the usual environments provided by 
% |amsmath|: definition, theorem, etc.
%    \begin{macrocode}
\newif\if@didactic@theorems
\@didactic@theoremstrue
\DeclareOptionX{notheorems}{\@didactic@theoremsfalse}
%    \end{macrocode}
% \end{option}
%
% \begin{option}{inner}
% \begin{option}{outer}
% \begin{option}{top}
% \begin{option}{bottom}
% We also have some key--value options to control the margins.
%    \begin{macrocode}
\newcommand\@didactic@margin@inner{20mm}
\newcommand\@didactic@margin@outer{60mm}
\newcommand\@didactic@margin@top{30mm}
\newcommand\@didactic@margin@bottom{60mm}
\DeclareOptionX{inner}[20mm]{\renewcommand\@didactic@margin@inner{#1}}
\DeclareOptionX{outer}[60mm]{\renewcommand\@didactic@margin@outer{#1}}
\DeclareOptionX{top}[25mm]{\renewcommand\@didactic@margin@top{#1}}
\DeclareOptionX{bottom}[60mm]{\renewcommand\@didactic@margin@bottom{#1}}
%    \end{macrocode}
% \end{option}
% \end{option}
% \end{option}
% \end{option}
%
% \begin{option}{marginparmargin}
% We also have an option to set the marginpar margin.
% We simply pass this on to |memoir| and keep a record ourselves too.
% We know that the default in |memoir| is |outer|.
%    \begin{macrocode}
\newcommand\@didactic@marginparmargin{outer}
\@ifclassloaded{memoir}{\marginparmargin{outer}}{}
\DeclareOptionX{marginparmargin}{%
  \renewcommand\@didactic@marginparmargin{#1}
  \@ifclassloaded{memoir}{%
    \marginparmargin{#1}
    \strictpagechecktrue
    \checkandfixthelayout
  }{}
}
%    \end{macrocode}
% \end{option}
%
% Now that we've declared the options we can process them.
%    \begin{macrocode}
\ProcessOptionsX\relax
%    \end{macrocode}
%
% \subsection{Customizing \texttt{beamer}}
%
% If |beamer| is loaded we want to customize it.
% However, only if the user hasn't disabled this with the |nobeamer| option.
%    \begin{macrocode}
\if@didactic@beamer
%    \end{macrocode}

% If we use |beamer| we want to use the Berlin theme.
% This theme guides the viewer very nicely, how must is left and where we are
% in the presentation.
% (The same is true for the presenter.)
%    \begin{macrocode}
\if@didactic@beamertheme
  \usetheme{Berlin}
%    \end{macrocode}
% However, we want to customize the footline.
% We want to add two lines in the footline.
% We want to add the author (left) and institute (right) to the first line.
% Then we add the title (left) and page number (right) to the bottom line.
%    \begin{macrocode}
  \setbeamertemplate{footline}%{miniframes theme}
  {%
    \begin{beamercolorbox}[colsep=1.5pt]{upper separation line foot}
    \end{beamercolorbox}
    \begin{beamercolorbox}[ht=2.5ex,dp=1.125ex,%
      leftskip=.3cm,rightskip=.3cm plus1fil]{author in head/foot}%
      \leavevmode{\usebeamerfont{author in head/foot}\insertshortauthor}%
      \hfill%
      {\usebeamerfont{institute in head/foot}%
        \usebeamercolor[fg]{institute in head/foot}\insertshortinstitute}%
    \end{beamercolorbox}%
    \begin{beamercolorbox}[ht=2.5ex,dp=1.125ex,%
      leftskip=.3cm,rightskip=.3cm plus1fil]{title in head/foot}%
      {\usebeamerfont{title in head/foot}\insertshorttitle}%
      \hfill%
      \insertframenumber%
    \end{beamercolorbox}%
    \begin{beamercolorbox}[colsep=1.5pt]{lower separation line foot}
    \end{beamercolorbox}
  }
%    \end{macrocode}
%
% We also want to set the transparency of the covered items.
% We also want to set the bibliography item to text, we need this to get proper 
% references (not icons) in the bibliography.
% (We need that for certain bibliography styles.)
%    \begin{macrocode}
  \setbeamercovered{transparent}
  \setbeamertemplate{bibliography item}{\relax}
\fi
%    \end{macrocode}
%
% Finally, we want to add a table of contents at the beginning of each section 
% and subsection.
% We want these to be shaded, so that the current section is highlighted.
% We also want to hide the subsections of the other sections.
%    \begin{macrocode}
\AtBeginSection[]{%
  \begin{frame}<beamer>
    \tableofcontents[currentsection,
                     subsectionstyle=show/show/hide,
                     subsubsectionstyle=show/show/hide]
  \end{frame}
}
\AtBeginSubsection[]{%
  \begin{frame}<beamer>
    \tableofcontents[currentsection,
                     subsectionstyle=show/shaded/hide,
                     subsubsectionstyle=show/show/hide]
  \end{frame}
}
\fi% end \if@didactic@beamer
%    \end{macrocode}
%
% \subsection{Customizing \texttt{memoir}}
%
% If |memoir| is loaded we want to customize it.
% However, only if the user hasn't disabled this with the |nomemoir| option.
%    \begin{macrocode}
\if@didactic@memoir
%    \end{macrocode}
%
% We want to set up |memoir| to use the Tufte style.
% This means that we want to put a lot of text in the margin.
% We'll use margin notes and put all the footnotes in the margin.
% We also want to use footnotes for references so that they also appear in the
% margin.
%
% We want to use the |marginfix| package to fix the margin notes.
%    \begin{macrocode}
\RequirePackage{marginfix}
\setlrmarginsandblock{\@didactic@margin@inner}
                     {\@didactic@margin@outer}
                     {*}
\setulmarginsandblock{\@didactic@margin@top}
                     {\@didactic@margin@bottom}
                     {*}

\footnotesinmargin

\RequirePackage{ragged2e}
\renewcommand{\sidefootform}{\RaggedRight}
\renewcommand{\foottextfont}{\footnotesize\RaggedRight}

\setmpjustification{\RaggedRight}{\RaggedRight}

% margin figure and caption typeset ragged against text block
\setfloatadjustment{marginfigure}{\mpjustification}
\setmarginfloatcaptionadjustment{figure}{\captionstyle{\mpjustification}}

% side captions
% https://tex.stackexchange.com/a/275626/17418
\sidecapmargin{outer}
\setsidecappos{t}
\checkandfixthelayout
\setsidecaps{\marginparsep}{\marginparwidth}
\renewcommand{\sidecapstyle}{%
  \captionstyle{\RaggedRight}
}

\makechapterstyle{arthangnum}{%
  \typeout{Using arthangnum chapter style.}
  \chapterstyle{article}
  \settowidth{\chapindent}{\chapnumfont 999}
  \renewcommand*{\printchaptername}{}
  \renewcommand*{\chapternamenum}{}
  \renewcommand*{\printchapternum}{%
    \noindent\llap{\makebox[\chapindent][l]{\chapnumfont \thechapter}}}
  \renewcommand*{\afterchapternum}{}
  \setsecheadstyle{\normalfont\large\bfseries}
  %\setsubsecheadstyle{\normalfont\normalsize\bfseries}
  \hangsecnum
}
% https://tex.stackexchange.com/a/33458/17418
\setsecnumdepth{subsection}
% https://tex.stackexchange.com/a/14354/17418
\maxtocdepth{subsection}

% https://tex.stackexchange.com/a/23041/17418
\RequirePackage{ifxetex}
\ifxetex
  \RequirePackage{fontspec}
  % https://tex.stackexchange.com/a/506776/17418
  \usepackage{newpxtext,newpxmath}
\else
  % https://tex.stackexchange.com/a/324757/17418
  % Palatino for main text and math
  \RequirePackage[osf,sc]{mathpazo}
  % Helvetica for sans serif
  % (scaled to match size of Palatino)
  \RequirePackage[scaled=0.90]{helvet}
  % Bera Mono for monospaced
  % (scaled to match size of Palatino)
  \RequirePackage[scaled=0.85]{beramono}
\fi
\setlxvchars\setxlvchars
\checkandfixthelayout
\nouppercaseheads
%    \end{macrocode}
%
% And that concludes the |memoir| part.
%    \begin{macrocode}
\fi% end \if@didactic@memoir
%    \end{macrocode}
%
% We also want to adapt the citation commands of a few packages to use 
% footnotes.
% For these, we check if the package is loaded, if it is, we do the changes.
% We don't load the packages ourselves, we assume the user has done that.
% This also means that the |didactic| package should be loaded last.
%    \begin{macrocode}
\@ifpackageloaded{biblatex}{%
  \ExecuteBibliographyOptions{%
    autocite=footnote,
    singletitle=false,
    maxbibnames=99,
    isbn=false,doi=false,url=false
  }
  % from https://tex.stackexchange.com/a/374059/17418
  \DeclareCiteCommand{\fullauthorcite}
    {\usebibmacro{prenote}}
    {\usedriver
      {\setcounter{maxnames}{99}% use up to 99 authors
        \DeclareNameAlias{sortname}{default}}
      {\thefield{entrytype}}}
    {\multicitedelim}
    {\usebibmacro{postnote}}
}{}
\@ifpackageloaded{csquotes}{%
  \SetCiteCommand{\autocite}
}{}
%    \end{macrocode}
%
% \subsection{A side-caption environment for \texttt{beamer}}
%
% \begin{environment}{sidecaption}
% When using |memoir|, we can use the |sidecaption| environment to put the 
% caption in the margin.
% However, when using |beamer| we don't have this environment, so we need to 
% provide it with a suitable behaviour.
% It's suitable to just translate it to a normal caption, as usually used in 
% |beamer|.
%    \begin{macrocode}
\ProvideDocumentEnvironment{sidecaption}{omo}{%
  \relax
}{
  \IfValueTF{#1}
    {\caption[#1]{#2\IfValueT{#3}{\label{#3}}}}
    {\caption{#2\IfValueT{#3}{\label{#3}}}}
}
%    \end{macrocode}
% \end{environment}
%
% \begin{macro}{\flushscap}
% We also want to provide a command to flush a figure towards the caption.
%    \begin{macrocode}
\NewDocumentCommand{\flushscap}{O{\centering}}{%
  \@ifclassloaded{memoir}{%
    \ifscapmargleft%
      \flushleft%
    \else%
      \flushright%
    \fi%
  }{%
    #1%
  }%
}
%    \end{macrocode}
% \end{macro}
%
%
% \subsection{More semantic environments}
%
% We want to provide a set of environments (blocks) for |beamer|.
% We want the names of the blocks to be translated.
% We'll use the |translations| package for this.
% That way we get the language used through the normal use of |babel|.
%
% We also want to have the same environments for the article mode, but in a 
% nicer format that the default of |beamerarticle|.
% For instance, instead of
% \begin{block}{Example}
% This is an example.
% \end{block}
% we want something like
% \begin{example}
% This is an example.
% \end{example}
%
% This means that we'll need to do one thing if |beamer| is loaded and another 
% if |beamerarticle| is loaded---or rather, when |beamer| is not loaded, we 
% should be able to use this without |beamer| and |beamerarticle|.
%
% We want to provide environments like this one:
% \begin{exercise}\label{exercise}
%   This is an exercise.
% \end{exercise}
% \begin{exercise}\label{anotherexercise}
%   This is another exercise.
% \end{exercise}
% We can use |\cref| to refer to them, getting something like \cref{exercise} or 
% \cref{exercise,anotherexercise}.
%
% We also want it to work with different languages, provided there is a
% translation.
% We do Swedish here.
%
% \selectlanguage{swedish}
% Vi kan också skapa exempel på svenska.
% \begin{exercise}\label{exempel}
%   Detta är ett exempel.
% \end{exercise}
% \begin{exercise}\label{ytterligareexempel}
%   Detta är ytterligare ett exempel.
% \end{exercise}
% Vi kan referera till dem med |\cref|, vilket ger något som \cref{exempel}
% eller \cref{exempel,ytterligareexempel}.
% \selectlanguage{british}
%
% Let's also add another idea, but in Swedish.
%
% \selectlanguage{swedish}
% \begin{idea}\label{ide}
%   Detta är en idé.
% \end{idea}
% Då har vi \cref{ide}, men vi hade även tidigare \cref{theidea}.
% Tillsammans är de \cref{ide,theidea}.
% \selectlanguage{british}
%
% We can also refer to them in English, \cref{ide} and \cref{theidea}
% separately and together as \cref{ide,theidea}.
%
% \begin{macro}{\ProvideSemanticEnv}
% We provide a command to create such environments.
% This way we just run this command in the preamble, if |beamer| is loaded it 
% creates the block environments for |beamer|, otherwise it creates the 
% environments for the article.
%
% It can be used like this:
% \begin{center}
% |\ProvideSemanticEnv{test}[alertblock]{Test}[style=definition]|\\
% \qquad|{test}{tests}{Test}{Tests}|
% \end{center}
%    \begin{macrocode}
\ProvideDocumentCommand{\ProvideSemanticEnv}{m o m o mmmm}{%
  \@ifundefined{#1}{%
    \@ifclassloaded{beamer}{% beamer
%    \end{macrocode}
% For |beamer| we want to use the |block| environment, or one of the coloured
% blocks that we provide below.
% We let the second argument (optional) be the name of the block environment to
% use.
% This will be the easiest way to set the colour of the block.
%
% Lastly, we let the third argument be the title of the block.
% This is the English title and also the key used to translate the title of the
% block.
%    \begin{macrocode}
      \IfValueTF{#2}{%
        \ProvideDocumentEnvironment{#1}{o}{%
          \begin{#2}{\GetTranslationWarn{#3}\IfValueT{##1}{: ##1}}
        }{%
          \end{#2}
        }
      }{%
        \ProvideDocumentEnvironment{#1}{o}{%
          \begin{block}{\GetTranslationWarn{#3}\IfValueT{##1}{: ##1}}
        }{%
          \end{block}
        }
      }
%    \end{macrocode}
%
% If we don't use |beamer|, we want to use the |thmtools| package to define the
% environments.
% The fourth argument (optional), will be passed as options to
% |\declaretheorem|.
% The fifth to eighth arguments are the keys of (meaning English) translations
% of the refnames of the block.
%    \begin{macrocode}
    }{% not beamer
      \IfValueTF{#4}{%
        \declaretheorem[
          name=\GetTranslationWarn{#3},
          refname={{\GetTranslationWarn{#5}},{\GetTranslationWarn{#6}}},
          Refname={{\GetTranslationWarn{#7}},{\GetTranslationWarn{#8}}},
          #4
        ]{#1}
      }{%
        \declaretheorem[
          style=definition,
          name=\GetTranslationWarn{#3},
          refname={{\GetTranslationWarn{#5}},{\GetTranslationWarn{#6}}},
          Refname={{\GetTranslationWarn{#7}},{\GetTranslationWarn{#8}}}
        ]{#1}
      }
    }
%    \end{macrocode}
%
% Since the |refname| and |Refname| options of |thmtools| doesn't seem to work
% we will add the necessary |\crefname| and |\Crefname| commands at the
% beginning of the document, for both |beamer| and non-|beamer| case.
%    \begin{macrocode}
    \AtBeginDocument{%
      \@ifpackageloaded{cleveref}{%
        \crefname{#1}
          {\GetTranslationWarn{#5}}
          {\GetTranslationWarn{#6}}
        \Crefname{#1}
          {\GetTranslationWarn{#7}}
          {\GetTranslationWarn{#8}}
      }{%
        \relax
      }
    }
  }{\relax} % \@ifundefined{#1}
} % \ProvideSemanticEnv
%    \end{macrocode}
% \end{macro}
% \begin{environment}{assumption}
% \begin{environment}{idea}
% \begin{environment}{question}
% \begin{environment}{exercise}
% \begin{environment}{activity}
% \begin{environment}{remark}
% \begin{environment}{summary}
% Then let's define some useful environments.
%    \begin{macrocode}
\ProvideSemanticEnv{assumption}[alertblock]{Assumption}
  {assumption}{assumptions}{Assumption}{Assumptions}
\ProvideSemanticEnv{idea}[greenblock]{Idea}
  {idea}{ideas}{Idea}{Ideas}
\ProvideSemanticEnv{question}[orangeblock]{Question}
  {question}{questions}{Question}{Questions}
\ProvideSemanticEnv{exercise}[yellowblock]{Exercise}
  {exercise}{exercises}{Exercise}{Exercises}
\ProvideSemanticEnv{activity}[yellowblock]{Activity}
  {activity}{activities}{Activity}{Activities}
\ProvideSemanticEnv{remark}[alertblock]{Remark}[%
    numbered=no,style=remark
  ]{remark}{remarks}{Remark}{Remarks}
\ProvideSemanticEnv{summary}[block]{Summary}[%
    numbered=no,style=remark
  ]{summary}{summaries}{Summary}{Summaries}
%    \end{macrocode}
% \end{environment}
% \end{environment}
% \end{environment}
% \end{environment}
% \end{environment}
% \end{environment}
% \end{environment}
%
% We also want to provide the normal environments for theorems and definitions
% and such.
%    \begin{macrocode}
\if@didactic@theorems
  \ProvideSemanticEnv{definition}[block]{Definition}
    {definition}{definitions}{Definition}{Definitions}
  \ProvideSemanticEnv{theorem}[block]{Theorem}[%
      numbered=unless unique,style=theorem
    ]{theorem}{theorems}{Theorem}{Theorems}
  \ProvideSemanticEnv{corollary}[block]{Corollary}[%
      numbered=unless unique,style=theorem
    ]{corollary}{corollaries}{Corollary}{Corollaries}
  \ProvideSemanticEnv{lemma}[block]{Lemma}[%
      numbered=unless unique,style=theorem
    ]{lemma}{lemmas}{Lemma}{Lemmas}
  \ProvideSemanticEnv{proof}[block]{Proof}[%
      numbered=no,style=proof
    ]{proof}{proofs}{Proof}{Proofs}
  \ProvideSemanticEnv{solution}[block]{Solution}[%
      numbered=no,style=proof
    ]{solution}{solutions}{Solution}{Solutions}
  \ProvideSemanticEnv{example}[exampleblock]{Example}
    {example}{examples}{Example}{Examples}
\fi
%    \end{macrocode}
%
%
% \subsection{Translations}
%
% We also want to provide translations for the environments.
% We want to provide those even if none of the packages above are loaded.
%    \begin{macrocode}
\ProvideTranslation{swedish}{Assumption}{Antagande}
\ProvideTranslation{swedish}{Assumptions}{Antagandena}
\ProvideTranslation{swedish}{assumption}{antagande}
\ProvideTranslation{swedish}{assumptions}{antagandena}
\ProvideTranslation{swedish}{Idea}{Idé}
\ProvideTranslation{swedish}{Ideas}{Idéerna}
\ProvideTranslation{swedish}{idea}{idé}
\ProvideTranslation{swedish}{ideas}{idéerna}
\ProvideTranslation{swedish}{Question}{Fråga}
\ProvideTranslation{swedish}{Questions}{Frågor}
\ProvideTranslation{swedish}{question}{fråga}
\ProvideTranslation{swedish}{questions}{frågor}
\ProvideTranslation{swedish}{Exercise}{Övningsuppgift}
\ProvideTranslation{swedish}{Exercises}{Övningsuppgifterna}
\ProvideTranslation{swedish}{exercise}{övningsuppgift}
\ProvideTranslation{swedish}{exercises}{övningsuppgifterna}
\ProvideTranslation{swedish}{Activity}{Aktivitet}
\ProvideTranslation{swedish}{Activities}{Aktiviteter}
\ProvideTranslation{swedish}{activity}{aktivitet}
\ProvideTranslation{swedish}{activities}{aktiviteter}
\ProvideTranslation{swedish}{Note}{Anmärkning}
\ProvideTranslation{swedish}{Notes}{Anmärkningar}
\ProvideTranslation{swedish}{note}{anmärkning}
\ProvideTranslation{swedish}{notes}{anmärkningar}
\ProvideTranslation{swedish}{Remark}{Anmärkning}
\ProvideTranslation{swedish}{Remarks}{Anmärkningar}
\ProvideTranslation{swedish}{remark}{anmärkning}
\ProvideTranslation{swedish}{remarks}{anmärkningar}
\ProvideTranslation{swedish}{Summary}{Sammanfattning}
\ProvideTranslation{swedish}{Summaries}{Sammanfattningar}
\ProvideTranslation{swedish}{summary}{sammanfattning}
\ProvideTranslation{swedish}{summaries}{sammanfattningar}
\ProvideTranslation{swedish}{definition}{definition}
\ProvideTranslation{swedish}{definitions}{definitionerna}
\ProvideTranslation{swedish}{Definition}{Definition}
\ProvideTranslation{swedish}{Definitions}{Definitionerna}
\ProvideTranslation{swedish}{theorem}{sats}
\ProvideTranslation{swedish}{theorems}{satserna}
\ProvideTranslation{swedish}{Theorem}{Sats}
\ProvideTranslation{swedish}{Theorems}{Satserna}
\ProvideTranslation{swedish}{corollary}{följdsats}
\ProvideTranslation{swedish}{Corollary}{Följdsats}
\ProvideTranslation{swedish}{corollaries}{följdsatser}
\ProvideTranslation{swedish}{Corollaries}{Följdsatser}
\ProvideTranslation{swedish}{lemma}{hjälpsats}
\ProvideTranslation{swedish}{lemmas}{hjälpsatserna}
\ProvideTranslation{swedish}{Lemma}{Hjälpsats}
\ProvideTranslation{swedish}{Lemmas}{Hjälpsatserna}
\ProvideTranslation{swedish}{proof}{bevis}
\ProvideTranslation{swedish}{Proof}{Bevis}
\ProvideTranslation{swedish}{proofs}{bevisen}
\ProvideTranslation{swedish}{Proofs}{Bevisen}
\ProvideTranslation{swedish}{Solution}{Lösningsförslag}
\ProvideTranslation{swedish}{Solutions}{Lösningsförslagen}
\ProvideTranslation{swedish}{solution}{lösningsförslag}
\ProvideTranslation{swedish}{solutions}{lösningsförslagen}
\ProvideTranslation{swedish}{Example}{Exempel}
\ProvideTranslation{swedish}{Examples}{Exemplen}
\ProvideTranslation{swedish}{example}{exempel}
\ProvideTranslation{swedish}{examples}{exemplen}
%    \end{macrocode}
%
%
% \subsection{Coloured blocks}
%
% \begin{environment}{lightblock}
% \begin{environment}{darkblock}
% \begin{environment}{blackblock}
% \begin{environment}{whiteblock}
% \begin{environment}{bwblock}
% \begin{environment}{redblock}
% \begin{environment}{blueblock}
% \begin{environment}{purpleblock}
% \begin{environment}{greenblock}
% \begin{environment}{yellowblock}
% \begin{environment}{orangeblock}
% Now we have the coloured blocks.
% These we want to define even if |beamer| is not loaded, but for this we need 
% |beamerarticle| instead.
%    \begin{macrocode}
\ProvideDocumentEnvironment{lightblock}{mm}{%
  \setbeamercolor{block body}{bg=#1!10,fg=black}
  \setbeamercolor{block title}{bg=#1,fg=black}
  \setbeamercolor{local structure}{fg=#1}
  \begin{block}{#2}
}{%
  \end{block}
}
\ProvideDocumentEnvironment{darkblock}{mm}{%
  \setbeamercolor{block body}{bg=#1!10,fg=black}
  \setbeamercolor{block title}{bg=#1,fg=white}
  \setbeamercolor{local structure}{fg=#1}
  \begin{block}{#2}
}{%
  \end{block}
}
\ProvideDocumentEnvironment{blackblock}{m}
{\begin{darkblock}{black}{#1}}
{\end{darkblock}}
\ProvideDocumentEnvironment{whiteblock}{m}{%
  \setbeamercolor{block body}{bg=white!10,fg=black}
  \setbeamercolor{block title}{bg=white,fg=black}
  \setbeamercolor{local structure}{fg=black}
  \begin{block}{#1}
}{%
  \end{block}
}
\ProvideDocumentEnvironment{bwblock}{m}{%
  \setbeamercolor{block body}{bg=white!10,fg=black}
  \setbeamercolor{block title}{bg=black,fg=white}
  \setbeamercolor{local structure}{fg=black}
  \begin{block}{#1}
}{%
  \end{block}
}
\ProvideDocumentEnvironment{redblock}{m}
{\begin{darkblock}{red}{#1}}
{\end{darkblock}}
\ProvideDocumentEnvironment{blueblock}{m}
{\begin{darkblock}{blue}{#1}}
{\end{darkblock}}
\ProvideDocumentEnvironment{purpleblock}{m}
{\begin{darkblock}{purple}{#1}}
{\end{darkblock}}
\ProvideDocumentEnvironment{greenblock}{m}
{\begin{lightblock}{green}{#1}}
{\end{lightblock}}
\ProvideDocumentEnvironment{yellowblock}{m}
{\begin{lightblock}{yellow}{#1}}
{\end{lightblock}}
\ProvideDocumentEnvironment{orangeblock}{m}
{\begin{lightblock}{orange}{#1}}
{\end{lightblock}}
%    \end{macrocode}
% \end{environment}
% \end{environment}
% \end{environment}
% \end{environment}
% \end{environment}
% \end{environment}
% \end{environment}
% \end{environment}
% \end{environment}
% \end{environment}
% \end{environment}
%
% \subsection{Adding margin notes of different kinds}
%
% We want to add titled margin notes.
% These margin notes should use indentation since they might contains several 
% paragraphs.
%
% \begin{macro}{\indentedmarginpar}
% We start with the indented margin notes.
% They simply take a text as argument and typeset it in the margin: 
% |\indentedmarginpar{this is the text}|.
% \indentedmarginpar{this is the text}
%    \begin{macrocode}
% Gives us indentation in the margin notes.
% Adapted from https://tex.stackexchange.com/a/257171
\RequirePackage{ragged2e}
\setlength{\RaggedRightParindent}{\parindent}
\NewDocumentCommand{\indentedmarginpar}{+m}{%
  \marginpar{%
    \footnotesize\RaggedRight
    \@afterindentfalse\@afterheading #1
  }%
}
%    \end{macrocode}
% \end{macro}
%
% \begin{macro}{\newnotetype}
% Now, let's add a command to add margin notes with a title.
% The |\newnotetype{\titlenote}{Title]| command creates a new command 
% |\titlenote| that takes a text as argument and typesets it in the margin with 
% the title |Title|: |\titlenote{this is the text}|.
% \newnotetype{\titlenote}{Title}
% \titlenote{this is the text}
%    \begin{macrocode}
\NewDocumentCommand{\newnotetype}{omm}{%
  \@ifclassloaded{beamer}{%
    \NewDocumentCommand{#2}{+m}{\relax}
  }{%
    \IfValueTF{#1}
      {\NewDocumentCommand{#2}{+m}{%
        \indentedmarginpar{#1{#3:} ##1}%
      }}
      {\NewDocumentCommand{#2}{+m}{%
        \indentedmarginpar{\emph{#3:} ##1}%
      }}%
  }
}
%    \end{macrocode}
% \end{macro}
%
% \begin{macro}{\NewNoteType}
% It can also be useful to be able to turn the notes off.
% For instance, we might want notes to appear in one version, but not in 
% others.
% Then we'd like a switch to simply turn them off.
% Also, we don't want these notes to appear in |beamer|, so we define them to 
% be off by default in that case.
%
% However, to achieve this, we must change the behaviour when creating new 
% notes: we need the name, not the |\command| as an argument.
% This is why we have the |\NewNoteType| command.
%
% \ltnote{This is a note with a title, using the note defined by the package.}
% We can use it like this:
% |\NewNoteType{lt}{Learning theory}|
% \NewNoteType{lt}{Learning theory}
% This creates a new note type |\lt|.
% |\lt{This is a note}| yields the note on the side.
% \lt{This is a note}
%
% We can also turn the notes off and on with |\ltoff| and |\lton|.
% \ltoff
% \lt{This is another note while off.}
% \lton
% \lt{This is another note while on, we didn't see the one we used while off.}
% \ltnote{This is a note with a title, using the note defined by the package.}
%
% Let's look at the implementation.
% We first create a new boolean for the note type.
% This will keep track of whether the note type is on or off, should be typeset 
% in the margin or if it does nothing.
% As mentioned above, we want to turn the notes off by default in |beamer|.
%    \begin{macrocode}
\NewDocumentCommand{\NewNoteType}{omm}{%
  \newbool{#2}
  \@ifclassloaded{beamer}
    {\boolfalse{#2}}
    {\booltrue{#2}}
%    \end{macrocode}
% Then we create the commands to turn the notes off and on.
% These must be created with |\expandafter\newcommand| to get the name of the
% note type.
% (|\NewDocumentCommand| can't accept names constructed with |\csname|.)
%    \begin{macrocode}
  \expandafter\newcommand\csname #2off\endcsname{\boolfalse{#2}}
  \expandafter\newcommand\csname #2on\endcsname{\booltrue{#2}}
%    \end{macrocode}
% We define the command differently depending on whether we want a custom
% formatting for the title of not.
% Same as before, we must use |\expandafter\newcommand| to construct these
% commands.
%    \begin{macrocode}
  \IfValueTF{#1}{%
    \expandafter\newcommand\csname #2\endcsname[1]{%
      \ifbool{#2}{%
        \indentedmarginpar{#1{#3:} ##1}%
      }{\relax}%
  }}{%
    \expandafter\newcommand\csname #2\endcsname[1]{%
      \ifbool{#2}{%
        \indentedmarginpar{\emph{#3:} ##1}%
      }{\relax}%
  }}%
}
%    \end{macrocode}
% \end{macro}
%
% \begin{macro}{\ltnote}
% We provide a few different types of margin notes.
% That way we can also include translations for them.
%    \begin{macrocode}
\ProvideTranslation{swedish}{Learning theory}{Lärandeteori}
\NewNoteType{ltnote}{\GetTranslationWarn{Learning theory}}
%    \end{macrocode}
% \end{macro}
%
%
% \subsection{Some useful PythonTeX functions}\label{PythonTeXdidactic}
%
% We want to provide some useful functions that can be used from PythonTeX.
% Particularly, we want it to generate code listings and run Python code and 
% then typeset them using |minted|.
% For instance, we want to be able to typeset the source code of a Python 
% function and then run it and typeset the output.
% We want to typeset the output from |pydoc| for a particular function or 
% module.
%
% We will create a few functions in PythonTeX, in the default session.
% These can then be used later.
% They are all prefixed with |didactic_| to avoid name clashes.
%
% The |minted| and |pythontex| environments can't be used in other functions.
% The design here is to work around that.
%    \begin{macrocode}
\begin{pycode}
import importlib.util
import inspect
import os
import pathlib
import re
import sys
import subprocess
import tempfile

def didactic_mint(output, /, lang="text",
                  minted_opts="escapeinside=||",
                  fullwidth=False, hlgrep=None):
  """
  Typeset the output using minted.

  - `lang` is the language of the output.
  - `minted_opts` are options to pass to minted. The default is to escape 
  inside the code using |...|.
  - `fullwidth` is a boolean to wrap the minted environment in a fullwidth 
  environment and blocks the margin. False by default.
  - `hlgrep` is a list of patterns to highlight in the output. Every line that 
  matches one of the patterns will be highlighted.
  """
  if fullwidth:
    print(r"\blockmargin")
    print(r"\begin{fullwidth}")

  if not minted_opts:
    minted_opts = ""
  if hlgrep:
    highlightlines = []
    for i, line in enumerate(output.splitlines()):
      for pattern in hlgrep:
        if re.search(pattern, line):
          highlightlines.append(i+1)
          break
    hl_opt = f",highlightlines={{{','.join(str(i) for i in highlightlines)}}}"
    if minted_opts:
      minted_opts += f",{hl_opt}"
    else:
      minted_opts = hl_opt

  with tempfile.NamedTemporaryFile(mode="w", delete=False,
                                   dir=".",
                                   prefix="didactic_output_",
                                   suffix=".txt") as f:
    f.write(output.strip())
    filename = f.name

  if minted_opts:
    print(r"\inputminted[%s]{%s}{%s}" % (minted_opts, lang, filename))
  else:
    print(r"\inputminted{%s}{%s}" % (lang, filename))

  if fullwidth:
    print(r"\end{fullwidth}")
    print(r"\unblockmargin")

def didactic_shell(cmd, /, shell=False, **kwargs):
  """
  Run a Python program and typeset the output using minted.

  - `cmd` is the command to run, it's a list, to be passed to `subprocess.run`.
  - `kwargs` are passed to `didactic_mint`, see that function for more 
  information.
  """
  output = subprocess.run(cmd,
                          shell=shell,
                          stdout=subprocess.PIPE,
                          stderr=subprocess.STDOUT)

  didactic_mint(output.stdout.decode().strip(), **kwargs)

def didactic_output(module, /, interpreter="python3", **kwargs):
  """
  Run a Python program and typeset the output using minted.

  - `module` is anything that can be passed as an argument to the Python 
  interpreter. Usually a file name or a module name.
  - `interpreter` is the Python interpreter to use. The default is `python3`.
  - `kwargs` are passed to `didactic_mint`, see that function for more 
  information.
  """
  didactic_shell([interpreter, module], **kwargs)

def didactic_pydoc(module, /, builtin=False, **kwargs):
  """
  Print the pydoc for a module.

  If `builtin` is True, print the pydoc for a builtin module. For non-built in 
  modules, pydoc adds a few lines of paths to the module. We don't want that, 
  so we cut those lines. For built-ins, however, it doesn't print those lines, 
  so we shouldn't cut any lines in that case.

  The `kwargs` are passed to `didactic_mint`, see that function for more 
  information.
  """
  output = subprocess.run(["python3", "-m", "pydoc", module],
                          capture_output=True)
  lines = output.stdout.decode().splitlines()
  if not builtin:
    lines = lines[:-4]
  didactic_mint("\n".join(lines).strip(), lang="text", **kwargs)

def didactic_help(module, /, **kwargs):
  """
  Print the help for a module.

  The `kwargs` are passed to `didactic_mint`, see that function for more 
  information.
  """
  didactic_mint(help(module), **kwargs)

def didactic_source(function, /, **kwargs):
  """
  Print the source code of a function. This is a function object and we use 
  inspection to get the source code of the actual Python function object.

  The `kwargs` are passed to `didactic_mint`, see that function for more 
  information.
  """
  didactic_mint(inspect.getsource(function).strip(), lang="python", **kwargs)

def didactic_import(module_name, module_path):
  """
  Import a module named `module_name` from a path `module_path`. This is useful 
  when we want to import a module that is not in the current directory. This 
  might be the case with PythonTeX, when it runs in a subdirectory and we want 
  to import a module that we're writing.
  """
  # from https://stackoverflow.com/a/67692/1305099
  spec = importlib.util.spec_from_file_location(module_name, module_path)
  the_module = importlib.util.module_from_spec(spec)
  sys.modules[module_name] = the_module
  spec.loader.exec_module(the_module)
  return the_module

def didactic_run_code(code, /, **kwargs):
  """
  Run a Python program and typeset the output using minted. The `code` is 
  Python code, not a file name. This function simply writes the code to a file 
  and then runs it using `didactic_output`.

  - `kwargs` are passed to `didactic_output`, which in turn passes some 
  arguments on to `didactic_mint`. See those functions for details.
  """
  with tempfile.NamedTemporaryFile(mode="w", delete=False,
                                   dir=".",
                                   prefix="didactic_code_",
                                   suffix=".py") as f:
    f.write(code)
    filename = f.name
  didactic_output(filename, **kwargs)
\end{pycode}
%    \end{macrocode}
%
%
% \subsection{Listing and running code examples}
%
% Let's turn our focus to |\lstexample| and |\runpython|.
% We want to make it easy to typeset and print the output of example Python 
% programs in slides and texts.
% We want to use PythonTeX to automatically run the code and typeset the source 
% code using |minted|.
%
% These are the same for both |beamer| and article mode.
% \begin{example}
%   Here's a code example:
%   \lstexample{python}{hello.py}
%   And here's the output:
%   \runpython{hello.py}
% \end{example}
%
% \begin{example}\label{ExampleSideBySide}
%   Here's another code example, set side by side with its output:
%   \textbytext*
%     {Code \hrulefill\par\lstexample{python}{hello.py}}
%     {Output \hrulefill\par\runpython{hello.py}}
% \end{example}
%
% \begin{macro}{\lstexample}
% The first part is easy.
% To typeset the source file we simply need to use the minted package.
% We provide an optional argument to pass options to minted.
% We also note that we need the |marginfix| package for the |\blockmargin| and
% |\unblockmargin| commands.
% We use these together with the |fullwidth| environment to typeset the code 
% using the additional space taken from the margin notes, hence we need to also 
% block the margin.
%    \begin{macrocode}
\RequirePackage{marginfix}
\NewDocumentCommand{\lstexample}{somm}{%
  \IfBooleanTF{#1}{\blockmargin\begin{fullwidth}}{\relax}
  \IfValueTF{#2}
    {\inputminted[escapeinside=||,#2]{#3}{#4}}
    {\inputminted[escapeinside=||]{#3}{#4}}
  \IfBooleanTF{#1}{\end{fullwidth}\unblockmargin}{\relax}
}
%    \end{macrocode}
% \end{macro}
%
% \begin{macro}{\runpython}
% Now, for the second part, printing the output, we do this in two steps.
% First, we create a function using PythonTeX that runs a Python program and 
% prints it output.
% We can use it like |\runpython{hello.py}| to run the program in |hello.py|.
% The output is printed in a |minted| environment.
% Alternatively we could use |\runpython[opt]{hello.py}| to pass |opt| to the 
% underlying |minted| environment.
%
% Let's create that function in PythonTeX to run the example program file.
% We simply execute it with Python 3 and capture its output.
% Then we print the output to stdout, which is then captured by PythonTeX.
%
% Then we simply create a command that run that function with the file as 
% argument.
% Then print the output in verbatim mode.
%    \begin{macrocode}
\NewDocumentCommand{\runpython}{som}{%
  \IfBooleanTF{#1}{%
    \pyc{fullwidth = True}
  }{
    \pyc{fullwidth = False}
  }
  \IfValueTF{#2}{%
    \pyc{minted_opts = '#2'}
  }{
    \pyc{minted_opts = None}
  }
  \pyc{didactic_output('#3', minted_opts=minted_opts, fullwidth=fullwidth)}
}
%    \end{macrocode}
% \end{macro}
% 
% \subsection{Contrasting code examples}
%
% Sometimes we want to contrast two code examples side by side.
% \begin{macro}{\textbytext}
% For example |\textbytext{\lipsum[1]}{\lipsum[1]}| should typeset the two 
% examples side by side.
% \ltnote{This is a note to test if it interferes with the text-by-text.}
% \textbytext{\lipsum[1]}{\lipsum[1]}
% 
% We can also use the starred version to use the full width.
% (This is in fact what we did in \cref{ExampleSideBySide} above.)
% In this case we use the space from the margin notes to expand into the 
% margin.
% \textbytext*{\lipsum[1]}{\lipsum[1]}
%
% This means that we can use the unstarred versions in floats.
% If we want to use the full width in the float, we expect the user to use the 
% |fullwidth| environment manually to set up the float properly.
%
% To do this we first add a |fullwidth| environment\footnote{%
%   Similarly to \url{https://tex.stackexchange.com/a/350944/17418}.
% }.
% We can get the |adjustwidth| environment from the |changepage| package 
% whenever |memoir| isn't used\footnote{%
%   The |memoir| class uses |changepage|, that's the origin of the package.
% }.
%    \begin{macrocode}
\RequirePackage{changepage}
\RequirePackage{calc}
\newlength{\@didactic@textbytext@oldcolumnwidth}
\NewDocumentEnvironment{fullwidth}{}{%
  \setlength{\columnwidth}{\textwidth+2em+\marginparwidth+\marginparsep}
  \if@twoside
    \IfStrEqCase{\@didactic@marginparmargin}{%
        {inner}{\begin{adjustwidth*}{-\marginparwidth-\marginparsep}{-2em}}
        {left}{\begin{adjustwidth}{-\marginparwidth-\marginparsep}{-2em}}
        {outer}{\begin{adjustwidth*}{-2em}{-\marginparwidth-\marginparsep}}
        {right}{\begin{adjustwidth}{-2em}{-\marginparwidth-\marginparsep}}
      }[\relax]
  \else
    \IfStrEqCase{\@didactic@marginparmargin}{%
        {inner}{\begin{adjustwidth}{-\marginparwidth-\marginparsep}{-2em}}
        {left}{\begin{adjustwidth}{-\marginparwidth-\marginparsep}{-2em}}
        {outer}{\begin{adjustwidth}{-2em}{-\marginparwidth-\marginparsep}}
        {right}{\begin{adjustwidth}{-2em}{-\marginparwidth-\marginparsep}}
      }[\relax]
  \fi
}{
  \if@twoside
    \end{adjustwidth*}
  \else
    \end{adjustwidth}
  \fi
}
%    \end{macrocode}
%
% Now to the |\textbytext| command.
% We simply set up a |tabular| environment\footnote{%
%   This requires the |array| package.
%   We can't use the |tabularx| package since that one expands the arguments 
%   several times to find a good width.
%   That doesn't go too well with |pythontex| and |minted|.
% } with two columns and use it inside a |fullwidth| environment.
% We note that if the optional star is used, we also block the margin and use 
% the full width.
% This can't be used inside a float.
%    \begin{macrocode}
\RequirePackage{array}
\NewDocumentCommand{\textbytext}{s+m+m}{%
  \IfBooleanTF{#1}
    {\blockmargin\begin{fullwidth}}
    {\begin{center}}
  \begin{tabular}{p{0.49\columnwidth}p{0.49\columnwidth}}
    #2 & #3
  \end{tabular}
  \IfBooleanTF{#1}
    {\end{fullwidth}\unblockmargin}
    {\end{center}}
}
%    \end{macrocode}
% \end{macro}
%
% Now, we could ask ourselves: why didn't we put that |\blockmargin| and 
% |\unblockmargin| inside the |fullwidth| environment?
% The reason is that we want to be able to use the |fullwidth| environment in 
% other cases, for instance, in figures.
% Floats block the margin by default, so we don't need to use it there.
% And even if we tried, it wouldn't work, since the |\blockmargin| and 
% |\unblockmargin| don't work in outer par mode (inside floats).
% (See the |marginfix| documentation for details.)
%
% \begin{macro}{\codebycode}
% When we deal with code, we deal with verbatim data.
% The easiest way to deal with this is to simply keep the code in files and 
% supply the file names to |\inputminted|.
% The |\codebycode| command is simply two |\inputminted| commands side by side:
% |\codebycode[opt1]{lang1}{file1}[opt2]{lang2}{file2}|.
% \codebycode
%   [breaklines,linenos]{python}{hello.py}
%   [breaklines,highlightlines=4]{python}{hello.py}
%
% This gives us the following implementation.
% We can reuse |\textbytext| to typeset the two examples side by side.
%    \begin{macrocode}
\RequirePackage{minted}
\NewDocumentCommand{\codebycode}{sommomm}{%
  \IfBooleanTF{#1}{\textbytext*}{\textbytext}
  {%
    \IfValueTF{#2}
      {\lstexample[#2]{#3}{#4}}
      {\lstexample{#3}{#4}}%
  }{%
    \IfValueTF{#5}
      {\lstexample[#5]{#6}{#7}}
      {\lstexample{#6}{#7}}%
  }%
}
%    \end{macrocode}
% \end{macro}
%
% \begin{macro}{\runbyrun}
% We also want to typeset the runs (|\runpython|):
% |\runbyrun[highlightlines=1]{hello.py}{hello.py}| yields
% \runbyrun[highlightlines=1]{hello.py}{hello.py}
% |\runbyrun{hello.py}[highlightlines=1]{hello.py}| yields
% \runbyrun{hello.py}[highlightlines=1]{hello.py}
% And, finally, |\runbyrun{hello.py}{hello.py}| yields
% \runbyrun{hello.py}{hello.py}
% We can use |\textbytext| to typeset the two outputs side by side.
%    \begin{macrocode}
\NewDocumentCommand{\runbyrun}{somom}{%
  \def\first{%
    \IfValueTF{#2}{\pyc{minted_opts_1 = '#2'}}{\pyc{minted_opts_1 = None}}
    \pyc{didactic_output('#3', minted_opts=minted_opts_1)}
  }
  \def\second{%
    \IfValueTF{#4}{\pyc{minted_opts_2 = '#4'}}{\pyc{minted_opts_2 = None}}
    \pyc{didactic_output('#5', minted_opts=minted_opts_2)}
  }
  \IfBooleanTF{#1}
    {\textbytext*{\first}{\second}}
    {\textbytext{\first}{\second}}
}
%    \end{macrocode}
% \end{macro}
%
% \Finale
\endinput
